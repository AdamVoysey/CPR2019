\documentclass[11pt]{article}
\usepackage{graphicx}
\author{C.~M.~Maynard and others}
\title{Report on LFRic computational performance 2019}
\date{\today}


\begin{document}
\maketitle
\medskip
\section{Introduction\label{sec:intro}}
{\bf THIS IS LAST YEAR's TEXT}
The LFRic model has undergone significant development in the past
year. Gung Ho science has also undergone development which has had a
positive impact on performance as well as imposing requirements on the
software infrastructure. The computational performance represents work
by the Gung Ho Dynamics, LFRic software and PSyclone development teams.

To be performant on potential, future, highly
parallel and diverse architectures, several features, algorithmic and computational
are required and this has led to the LFRic design and performance
strategy. The separation of concerns of the Natural and Computational
Science is a key design feature which allows different parallel models
to be employed to target different architectures by automatic code generation.

However, several problems with libraries used by the LFRic models and
the compilers themselves have inhibited progress. The ESMF library has
been used for the distributed memory structures and communications
over MPI. The ESMF library has a 32-bit limit for integers. This
creates a problem for LFRic as partitioning the mesh and determining
the routing tables for remote communication requires the global mesh
to be held in local memory whilst the initialisation is
performed. This has limited the problem size that has been able to
used and thus a strong scaling limit. A modification of to the
high-order coordinate field for which routing tables are not required
has allowed LFRic to scale to a cubed sphere C1944. Where this denotes
$1944 \times 1944$ cells per panel of the cube. This has a global
model resolution of 5km, with 30 levels. 

The LFRic project has a mitigation strategy in place which will allow
the routing table calculation to be performed more efficiently, both
in execution time and memory capacity which will allow LFRic to be
able to use a 1km global mesh. A second mitigation strategy is also in
place to use the Library YAXT, instead of ESMF.  Whilst YAXT initially
has similar problems to ESMF, these are much simpler to resolve as the
YAXT is significantly smaller than ESMF and simpler for outside
developers (such as the LFRic team) to modify. 

There have been several issues with compilers. Both the Intel and PGI
compilers have had problems compiling legal Fortran 2003 Object
Oriented code. The Intel problems have been mitigated by work-arounds
and even be fixed in the 2018 release of the compiler. The Cray
compiler produces a run-time memory error from the ESMF library when
run in parallel. It is believed (by Cray at least) that this issue is
now fixed. The PGI compiler has problems and these are under active
investigation with the compiler group. All these issues have consumed
a non-trivial amount of time on behalf of the LFRic and have inhibited
the investigation and implementation of performance.

The report is divided into two sections, the solver and Scaling
Analysis and single socket computational performance.

\section{Independent section}
Code runs great etc.

\end{document}

