\documentclass[11pt]{article}
\usepackage{graphicx}
\usepackage{cite}
\author{S.~A.~Adams, S.~Cusworth, C.~M.~Maynard and others}
\title{Report on LFRic computational performance 2019}
\date{\today}


\begin{document}
\maketitle
\medskip
\section{Introduction\label{sec:intro}}
The LFRic infrastructure is designed to host a dynamical core (Gung Ho)
that is scalable to a very large degree across a distributed memory
computer. This is typically expressed over MPI. LFRic is also designed
to accommodate different programming models to target different
processor architectures. Currently, OpenMP 2.0 is supported for shared
memory parallelism on CPUs. This is being extended to support OpenACC
to target both shared memory parallelism and Instruction Level
Parallelism (IPL) on GPUs. Moreover, the infrastructure has been
developed to allow the infrastructure to support import features such
as I/O using the IO server pattern in the XIOS library. Other
developments include the abstract solver API which allows for great
flexibility in constructing different solvers, redundant computation
aalgorithms into halo regions which boost the efficiency of shared
memory and reduce the amount of communication and finally a Multigrid
solver which will enable a large reduction in the cost of global communication.
Much of the model infrastructure is described in~\cite{LFRic}.

\section{Independent section}
Code runs great etc.

\bibliography{refs.bib}
\bibliographystyle{unsrt}

\end{document}

