\documentclass[11pt]{article}
\usepackage{graphicx}
\usepackage{cite}
\author{S.~A.~Adams, S.~J.~Cusworth, C.~M.~Maynard and others}
\title{Report on LFRic computational performance 2019}
\date{\today}


\begin{document}
\maketitle
\medskip
\section{Introduction\label{sec:intro}}
The LFRic infrastructure is designed to host a dynamical core (Gung Ho)
that is scalable to a very large degree across a distributed memory
computer. This is typically expressed over MPI. LFRic is also designed
to accommodate different programming models to target different
processor architectures. Currently, OpenMP 2.0 is supported for shared
memory parallelism on CPUs. This is being extended to support OpenACC
to target both shared memory parallelism and Instruction Level
Parallelism (IPL) on GPUs. Moreover, the infrastructure has been
developed to allow the infrastructure to support import features such
as I/O using the IO server pattern in the XIOS library. Other
developments include the abstract solver API which allows for great
flexibility in constructing different solvers, redundant computation
algorithms into halo regions which boost the efficiency of shared
memory and reduce the amount of communication and finally a Multigrid
solver which will enable a large reduction in the cost of global communication.
Much of the model infrastructure is described in~\cite{LFRic}.

The LFRic model uses PSyclone to generate the Parallel System or PSy
layer. Here, data parallelism across the horizontal mesh is used,
exploiting both distributed and shared memory parallelism. This
exploits domain specific information, known to the science developer
and encoded as Metadata in the kernel layer. However, to
fully exploit the ILP on more parallel architectures such as GPUs, it
is necessary to extend PSyclone to fully parse individual Fortran
statements in the Kernel layer and then, for example, annotate the
source code with directives such as OpenACC or OpenMP. This work is just beginning,
however, the micro-benchmark~\cite{lfric-microbenchmarks} suite has been developed so computational
intensive kernels can be used to explore what kernel optimisation can
be developed which can then be generated by PSyclone in the full
model. The PSyclone Kernel Extractor (PSKE) is being developed to
allow the automated extraction of kernels, or more specifically a
driver with a dump file containing the necessary data for the looping
over the horizontal mesh and degrees of freedom without any of the
LFRic infrastructure.

LFRic uses Fortran 2003 Object Orientation programming to support many of the
features described above. This is a very powerful programming
style which aids software development by allowing a clear separation
of concerns between different areas of the code, promoting code re-use
and code development by disentangling dependencies. However, the lack
of compiler support, or more precisely, the proliferation of compiler
bugs which prevent the use of particular compilers, either without specific
code work arounds or even at all, is major problem. Development has
been severely delayed whilst compiler bugs have been isolated and
reported and work-arounds sought. The lack of working Fortran 2003
compilers remains a source of concern and must be considered a risk to
the project.

The strategy to obtain compute performance for LFRic has four main
strands. IO performance is achieved by using the asynchronous IO sever
pattern to hide the code of writing data, this is achieved by using
the XIOS library. Compute performance is achieved by enabling
scalability, allowing many agents to do computation. This comes at the
cost of communication. Local communication in the form of halo
exchange is reduced by employing a communication avoiding/reducing
algorithm of redundant computation into the halos and exploiting shared
memory threaded parallelism. Global communications are primarily in
the form of global sums required for iterative solvers. The global sum
is already vendor optimised both in the vendor supplied MPI library
implementation and the high-bandwidth, low-latency network of the
machine itself. The only way to reduce this cost and ultimate limit on
scalability is to use an algorithm with less global sums. This is the
Geometric multigrid pre-conditioner to the Helmholtz solver. The {\em
  on-node} compute performance of the LFRic code itself is examined by
considering the performance of computationally intensive kernels on
different processor architectures. The LFRic Microbenchmark suite can
be used to develop architecture specific optimisations.
The report is organised into sections on each of these four strands,
{\em viz.} IO, local communication, global communication and
architecture specific optimisations.


\section{Independent section}
Code runs great etc.

\section{CMA micro-benchmark on Broadwell and TX2}
In this section we will introduce the CMA\_appinv micro-benchmark and present performance results from tests on the Intel Broadwell and Cavium ARM-based ThunderX2 architectures.
In addition we compare performance of the micro-benchmark when built with different compilers and runtime libraries.

\subsection{CMA micro-benchmark}

The LFRic dynamical model performs operations to solve
\begin{equation} \label{eq:matvec}
A \cdot \mathbf{x} = \mathbf{b}
\end{equation}
in various guises and contexts where $A$ is a matrix representing a partial differential equation, $\mathbf{b}$ is a vector containing information describing boundary conditions and $\mathbf{x}$ is a state vector.
The finite element methods that the GungHo application uses are written to perform operations on a cell-by-cell basis in what is termed Local Matrix Assembly (LMA).
The LMA micro-benchmark discussed in the previous report is a characteristic example of such methods.
The LFRic infrastructure also contains methods for solving equation \ref{eq:matvec} on a column-by-column basis.
Although these Column Matrix Assembly (CMA) methods have yet to be utilised by the full GungHo application they are deployed in the gravity\_wave mini-app.

For this report we generated a new CMA micro-benchmark using LFRic revision 15378 focused on the columnwise\_op\_appinv\_kernel\_code() routine in the gravity\_wave mini-app.
This kernel is used in the pre-conditioner for the Helmholtz solver.
As per the LMA micro-benchmark described in the previous report, the driver code was extracted from the Psyclone-generated Psy-layer and the kernel was taken directly from the LFRic source code.
By design, the driver routine contains the so called global optimisations, namely OpenMP worksharing and colouring (to prevent race conditions) over columns on the sphere.

The dinodump capability from PYSKE was used to generate Known Good Output (KGO) which we use to verify the micro-benchmark behaved as expected in later tests.
This involved adding additional PSYKE subroutine calls to the Psy-layer code in the gravity\_wave mini-app in order to capture the input scalars and arrays for the kernel.
The micro-benchmark driver code was also trivially modified to read in the produced dinodump before calling the kernel and compare the result of the kernel against the KGO.
Before running the mini-app to generate the dinodumo the number of vertical layers in the configuration was modified from the default (10) to 256.
While current NWP and climate GA and RA production configurations use between 70 and 90 vertical (UM) levels we took 256 layers in order to increase the throughput of the benchmark and also in expectation that future configurations will run with increased vertical resolution.


While generating the new micro-benchmark a minor bug in kernel was discovered.
Specifically, the manner in which the kernel is called in gravity\_wave mini-app (and the supporting code comments) indicate that the kernel is called to update the target array but instead the kernel actually populates the array while overriding the initial contents.
It was confirmed by the original author of the code that the functionality of the kernel was correct and the comments are in error.
The bug has minimal performance impact on CPU architectures- an intent(inout) array should be intent(out)- but may have a larger impact on GPU architectures where the array would be being offloaded unnecessarily. 

The resultant micro-benchmark can be found in the micro-benchmark ~\cite{lfric-microbenchmarks} suite on github.
We note at this point when making comparisons between the LMA and CMA\_appinv micro-benchmarks we should be mindful that they serve different purposes in the solver code.
Specifically, the LMA micro-benchmark applies the matrix $A$ to a vector $\mathbf{x}$ whereas the CMA\_inv micro-benchmark applies the inverse matrix $A^{-1}$ to $\mathbf{b}$, in other words it solves equation \ref{eq:matvec} for $\mathbf{x}$.

\subsection{CMA\_appinv performance}

We present timings for the kernel on Intel Broadwell and Cavium ThunderX2 as measured using omp\_get\_wtime().
In each case the kernel was run for 1000 iterations in order to obtain a smooth out runtime variability.
% Timing the routine in this way may however slightly exaggerate the time spend performing thread synchronisation, a cost we estimate to be within the noise of the system.
We use the latest stable version of each compiler available on the two platforms at the time of writing.
Each executable was built with the -O3 optimisation flag.

\subsubsection{Intel Broadwell}
On the Broadwell platform used (XCS) the compilers used were
\begin{itemize}
\item CCE 8.7.7
\item Intel 17.0.0.098
\item GNU 7.3.0
\end{itemize}



During the development of this work we encountered notable differences in performance between compiler versions.
For example, a measurable performance degradation between cce 8.6.0 and the cce 8.7.x series.
The single threaded (OMP\_NUM\_THREADS=1) runtime increased from 7.4s to 10.5s when changing between the two versions.
In other words cce 8.6.0 produces a faster executable than the Intel compiler but the 8.7.7 executable is slower than the Intel generated version.
This discrepancy is outwith the observed run-to-run variability on the system and is unexplained at present.
We use the more recent (but slightly less performant) cce 8.7.7.

\subsubsection{TX2}
Isambard is a XC50 system hosted by the Met Office and funded by the GW4 alliance and EPSRC.
The heterogeneous system contains chips of various architectures including 164 Cavium ThunderX2 nodes.
At the time of writing (after the March 2019 upgrade to CLE7.0) the supported compilers on the ARM-based nodes are
\begin{itemize}
\item CCE 8.7.9
\item GNU 8.2.0
\item Allinea 19.0.1
\end{itemize}
where the Allinea compiler is built on top of LLVM and Flang.



\subsection{CMA\_appinv performance using drop-in library call}
Here we investigate whether the implementation of the Thomas algorithm as written in the kernel can be replaced with a call to a standard library- in this case LAPACK.
The potential benefits of this include a potential improvement to runtimes of current architectures.
But, more importantly calling a standard API allows us to tap into routines that have been optimised specifically for the architecture being used.
This could provide performance portability while reducing porting and maintenance costs when moving between future compute platforms.
A potential problem with this plan is that the memory layout used in the columnwise operator type objects differs from that expected by the LAPACK API.

The version of the micro-benchmark used can be found in a branch of the github repo.

Thread scaling
Thread-scales OK but runtime is poor.
Is it memory copies? No, we tried taking these out but couldn't match ``vanilla'' performance.

\subsection{Discussion}

It is worth noting that there are few opportunities for vectorisation in the current implementation.

In avoiding building the LFRic infrastructure we also avoid compiler support problems.
TX2 vs Broadwell
Single thread performance on Broadwell is comfortably better than TX2. However, when multithreading is considered then the machines are node-for-node comparable.
Use of drop-in lib
Use of LAPACK as a drop-in replacement for kernel. Not optimal at present- LAPACK is optimised for larger problems.

Worth investigating vectorised versions of the algorithm, especially in the context of GPUs.


\bibliography{refs.bib}
\bibliographystyle{unsrt}

\end{document}

